\documentclass{article}

% The package amsmath defines the environment equation*
\usepackage{amsmath}
% The package amssymb adds the macro \mathbb{...}
\usepackage{amssymb}
% The package amssymb adds the macro \degree
\usepackage{gensymb}
\usepackage{multicolrule}
\columnseprule=0.5pt
\usepackage{parskip}
% The package geometry allows to reduce the margin
\usepackage[
    a2paper,
    top=2.5cm,
    left=0.5cm,
    right=0.5cm,
    bottom=2.5cm
]{geometry}

\usepackage{titlesec}
\titleformat{\section}  % which section command to format
  {\fontsize{20}{20}\bfseries} % format for whole line
  {\thesection} % how to show number
  {1em} % space between number and text
  {} % formatting for just the text
  [] % formatting for after the text

\titleformat{\subsection}
  {\fontsize{18}{18}\bfseries} % format for whole line
  {\thesubsection} % how to show number
  {1em} % space between number and text
  {} % formatting for just the text
  [] % formatting for after the text

\usepackage{xcolor}
\usepackage{tikz}
\usetikzlibrary{arrows.meta}

% +------------------------------+
% |                              |
% | Borde alrededor de la página |
% |                              |
% +------------------------------+

\usepackage{scrlayer-scrpage}
\newcommand{\jamanta}{\tikz[remember picture,overlay]
    \draw [black,line width=2mm]
    (current page.south west)
    rectangle
    (current page.north east)
    ;}
% \chead stands for "Center Header". It is a command provided by the
% scrlayer-scrpage package used to define what appears in the
% top-middle portion of the page.
\chead[\jamanta]{\jamanta}

% +----------------------------+
% |                            |
% | Encabezado y pie de página |
% |                            |
% +----------------------------+

% Contenido a la izquierda en el encabezado
\ihead{\fontsize{30pt}{36pt}\selectfont Identidades trigonométricas}
% Contenido a la izquierda en el pie de página
\ifoot{\fontsize{30pt}{48pt}\selectfont Left Footer}
% Contenido a la derecha en el pie de página
\ofoot{\fontsize{30pt}{48pt}\selectfont Hecho con \LaTeX}

\setheadsepline{1pt} % Ancho de la línea horizontal bajo el encabezado de la página
\setfootsepline{1pt} % Ancho de la línea horizontal arriba del pie de página

% +-----------------------+
% |                       |
% | Símbolos customizados |
% |                       |
% +-----------------------+

\newsavebox \myequalbox
\sbox \myequalbox {%
    \begin{tikzpicture}[scale=0.2, baseline]
        \coordinate (A) at (0,0);
        \coordinate (B) at (1.5,0);
        \coordinate (C) at (0,0.6);
        \coordinate (D) at (1.5,0.6);
        \draw[blue, line width = 1.5pt] (A) -- (B);
        \draw[blue, line width = 1.5pt] (C) -- (D);
    \end{tikzpicture}%
}

\newcommand\myequal{\mathrel{\vcenter{\hbox{\usebox\myequalbox}}}}

\newsavebox\myrightarrowbox
\sbox\myrightarrowbox{%
    \begin{tikzpicture}[baseline=-0.7ex] % Adjusts vertical alignment
        \draw[red, line width = 1pt, -{Computer Modern Rightarrow[length=6pt, width=8pt]}] (0,0) -- (0.5,0);
    \end{tikzpicture}%
}
\newcommand\myrightarrow{\mathrel{\vcenter{\hbox{\usebox\myrightarrowbox}}}}

\newsavebox\myleftrightarrowbox
\sbox\myleftrightarrowbox{%
    \begin{tikzpicture}[baseline=-0.7ex] % Adjusts vertical alignment
        \draw[red, line width = 1pt, {Computer Modern Rightarrow[length=6pt, width=8pt]}-{Computer Modern Rightarrow[length=6pt, width=8pt]}] (0,0) -- (0.6,0);
    \end{tikzpicture}%
}
\newcommand\myleftrightarrow{\mathrel{\vcenter{\hbox{\usebox\myleftrightarrowbox}}}}

\newsavebox\mylandbox
\sbox\mylandbox{%
    \begin{tikzpicture}
        \draw[orange, line width = 1.5pt] (0,0) -- (0.15, 0.30) -- (0.3, 0);
    \end{tikzpicture}%
}

\newcommand\myland{\mathbin{\vcenter{\hbox{\usebox\mylandbox}}}}

% +--------------------------------------------------
% |
% | Inicio del documento
% |
% +--------------------------------------------------

\begin{document}


\fontsize{14pt}{14pt}\selectfont

\begin{multicols}{3}
\section{Identidades trigonométricas de ángulos simples}

\textbf{Identidades de las razones trigonométricas recíprocas}

\begin{itemize}
    \item $\sin\left(x\right) \cdot \csc\left(x\right) \myequal 1 \myrightarrow \csc\left(x\right) = \dfrac{1}{\sin\left(x\right)}$
    \item $\cos\left(x\right) \cdot \sec\left(x\right) \myequal 1 \myrightarrow \sec\left(x\right) = \dfrac{1}{\cos\left(x\right)}$
    \item $\tan\left(x\right) \cdot \cot\left(x\right) \myequal 1 \myrightarrow \cot\left(x\right) = \dfrac{1}{\tan\left(x\right)}$
\end{itemize}

\textbf{Identidades por división o por cociente}

\begin{itemize}
    \item $\tan\left(x\right) \myequal \dfrac{\sin\left(x\right)}{\cos\left(x\right)}$
    \item $\cot\left(x\right) \myequal \dfrac{\cos\left(x\right)}{\sin\left(x\right)}$
\end{itemize}

\textbf{Identidades pitagóricas}

\begin{itemize}
    \item $\sin^2\left(x\right) + \cos^2\left(x\right) \myequal 1$
    \item $\sec^2\left(x\right) \myequal 1 + \tan^2\left(x\right)$
    \item $\csc^2\left(x\right) \myequal 1 + \cot^2\left(x\right)$
\end{itemize}

\textbf{Identidades auxiliares}

\begin{itemize}
    \item $\left( \sin\left(x\right) \pm \cos\left(x\right) \right)^2 \myequal 1 \pm 2 \sin\left(x\right) \cos\left(x\right)$
    \item $\sin^4\left(x\right) + \cos^4\left(x\right) \myequal 1 - 2 \sin^2\left(x\right) \cos^2\left(x\right)$
    \item $\sin^6\left(x\right) + \cos^6\left(x\right) \myequal 1 - 3 \sin^2\left(x\right) \cos^2\left(x\right)$
    \item $\tan\left(x\right) + \cot\left(x\right) \myequal \sec\left(x\right) \csc\left(x\right)$
    \item $\sec^2 \left(x\right) + \csc^2 \left(x\right) \myequal \sec^2\left(x\right) \csc^2 \left(x\right)$
    \item $\left( 1 \pm \sin\left(x\right) \pm \cos\left(x\right) \right)^2 \myequal 2 \left(1 \pm \sin\left(x\right)\right) \left(1 \pm \cos\left(x\right)\right)$
    \item $\sec\left(x\right) + \tan\left(x\right) \myequal \dfrac{1}{\sec\left(x\right) - \tan\left(x\right)}$
    \item $\csc\left(x\right) + \cot\left(x\right) \myequal \dfrac{1}{\csc\left(x\right) - \cot\left(x\right)}$
    \item $a \cdot \sin\left(x\right) + b \cos\left(x\right) = \sqrt{a^2 + b^2} \\ \myrightarrow  \sin\left(x\right) \myequal \dfrac{a}{\sqrt{a^2 + b^2}} \myland \cos\left(x\right) \myequal \dfrac{b}{\sqrt{a^2 + b^2}}$
    \item $\dfrac{\cos\left(x\right)}{1 \pm \sin\left(x\right)} \myequal \dfrac{1\mp\sin\left(x\right)}{\cos\left(x\right)}$
    \item $\dfrac{\sin\left(x\right)}{1 \pm \cos\left(x\right)} \myequal \dfrac{1\mp\cos\left(x\right)}{\sin\left(x\right)}$
\end{itemize}

\textbf{Desigualdades}

\begin{itemize}
    \item $\forall x \in \mathbb{R} , n \in \mathbb{Z}^{+} \colon \dfrac{1}{2^{n-1}} \leq \sin^{2n} \left(x\right) + \cos^{2n} \left(x\right) \leq 1$
    \item $\forall x, a, b\in \mathbb{R} \colon - \sqrt{a^2 + b^2} \leq a \cdot \sin\left(x\right) + b \cdot \cos\left(x\right) \leq \sqrt{a^2 + b^2}$
    \item $\forall a, b\in \mathbb{R}^{+} \colon \tan\left(x\right) > 0 \myrightarrow a \cdot \tan\left(x\right) + b \cdot \cot\left(x\right) \geq 2 \sqrt{ab}$
\end{itemize}

\section{Identidades trigonométricas de ángulos compuestos}

\textbf{Identidades de la suma y diferencia de dos ángulos}

\begin{itemize}
    \item $\sin \left(x \pm y\right) \myequal \sin\left(x\right) \cos\left(y\right) \pm \cos\left(x\right) \sin\left(y\right)$
    \item $\cos \left(x \pm y\right) \myequal \cos\left(x\right) \cos\left(y\right) \mp \sin\left(x\right) \sin\left(y\right)$
    \item $\tan \left(x \pm y\right) \myequal \dfrac{\tan\left(x\right)\pm \tan\left(y\right)}{1 \mp \tan\left(x\right) \tan\left(y\right)}$
\end{itemize}

\textbf{Identidades auxiliares de la suma y diferencia de dos ángulos}

\begin{itemize}
     \item $\sin\left(x + y\right) \cdot \sin\left(x - y\right) \myequal \sin^2\left(x\right) - \sin^2\left(y\right)$
     \item $\cos\left(x + y\right) \cdot \cos\left(x - y\right) \myequal \cos^2\left(x\right) - \sin^2\left(y\right)$
     \item $\tan\left(\theta\right) \myequal \dfrac{b}{a} \myleftrightarrow a \cdot \sin\left(x\right) \pm b \cdot \cos\left(x\right) \myequal \sqrt{a^2 + b^2} \cdot \sin\left(x \pm \theta\right)$
     \item $\tan\left(x \pm y\right) \myequal \tan\left(x\right) \pm \tan\left(y\right) \pm \tan\left(x\right) \tan\left(y\right) \tan\left(x \pm y\right)$
\end{itemize}

\subsection{Identidades para tres ángulos}

Si $\exists k \in \mathbb{Z} \colon x + y + z = k \pi$, entonces:

\begin{itemize}
    \item $\tan\left(x\right) + \tan\left(y\right) + \tan\left(z\right) \myequal \tan\left(x\right) \tan\left(y\right) \tan\left(z\right)$
    \item $\cot\left(x\right) \cot\left(y\right) + \cot\left(y\right) \cot\left(z\right) + \cot\left(z\right) \cot\left(x\right) \myequal 1$
\end{itemize}

Si $\exists k \in \mathbb{Z} \colon x + y + z = \displaystyle \left(2k+1\right)\dfrac{\pi}{2}$, entonces:

\begin{itemize}
    \item $\cot\left(x\right) + \cot\left(y\right) + \cot\left(z\right) \myequal \cot\left(x\right) \cot\left(y\right) \cot\left(z\right)$
    \item $\tan\left(x\right) \tan\left(y\right) + \tan\left(y\right) \tan\left(z\right) + \tan\left(z\right) \tan\left(x\right) \myequal 1$
\end{itemize}

\section{Identidades trigonométricas de múltiples ángulos}

\subsection{Identidades trigonométricas del ángulo doble}

\begin{itemize}
    \item $\sin\left(2x\right) \myequal 2 \sin\left(x\right) \cos\left(x\right)$
    \item $\cos\left(2x\right) \myequal \cos^2\left(x\right) - \sin^2\left(x\right)$
    \item $\tan\left(2x\right) \myequal \dfrac{2 \tan\left(x\right)}{1 - \tan^2\left(x\right)}$
    \item $\sin\left(2x\right) \myequal \dfrac{2 \tan\left(x\right)}{1 + \tan^2\left(x\right)}$
    \item $\cos\left(2x\right) \myequal \dfrac{1 - \tan^2\left(x\right)}{1 + \tan^2\left(x\right)}$
    \item $\cos\left(2x\right) \myequal 1 - 2 \sin^2\left(x\right) $
    \item $\cos\left(2x\right) \myequal 2 \cos^2\left(x\right) - 1$
\end{itemize}

\textbf{Identidades de degradación}

\begin{itemize}
    \item $2 \sin^2\left(x\right) \myequal 1 - \cos\left(2x\right)$
    \item $2 \cos^2\left(x\right) \myequal 1 + \cos\left(2x\right)$
    \item $8 \sin^4\left(x\right) \myequal 3 - 4 \cos\left(2x\right) + \cos\left(4x\right)$
    \item $8 \cos^4\left(x\right) \myequal 3 + 4 \cos\left(2x\right) + \cos\left(4x\right)$
    \item $\sin^4\left(x\right) + \cos^4\left(x\right) \myequal \dfrac{3}{4} + \dfrac{1}{4} \cos\left(4x\right)$
    \item $\sin^6\left(x\right) + \cos^6\left(x\right) \myequal \dfrac{5}{8} + \dfrac{3}{8} \cos\left(4x\right)$
\end{itemize}

\textbf{Identidades auxiliares}

\begin{itemize}
    \item $\left(\sin\left(x\right) \pm \cos\left(x\right)\right)^2 \myequal 1 \pm \sin\left(2x\right)$
    \item $\sqrt{1 \pm \sin\left(2x\right)} \myequal \left|\sin\left(x\right) \pm \cos\left(x\right)\right|$
    \item $\cot\left(x\right) + \tan\left(x\right) \myequal 2 \csc\left(2x\right)$
    \item $\cot\left(x\right) - \tan\left(x\right) \myequal 2 \cot\left(2x\right)$
    \item $\tan\left(2x\right) \cdot \cot\left(x\right) \myequal \sec\left(2x\right) + 1$
    \item $\tan\left(2x\right) \cdot \tan\left(x\right) \myequal \sec\left(2x\right) - 1$
\end{itemize}

\subsection{Identidades trigonométricas del ángulo mitad}

\begin{itemize}
    \item $\sin\left(\dfrac{x}{2}\right) \myequal \pm \sqrt{\dfrac{1 - \cos\left(x\right)}{2}}$
    \item $\cos\left(\dfrac{x}{2}\right) \myequal \pm \sqrt{\dfrac{1 + \cos\left(x\right)}{2}}$
    \item $\tan\left(\dfrac{x}{2}\right) \myequal \pm \sqrt{\dfrac{1 - \cos\left(x\right)}{1 + \cos\left(x\right)}}$
\end{itemize}

El signo de $\sin\left(\frac{x}{2}\right)$, $\cos\left(\frac{x}{2}\right)$ y $\tan\left(\frac{x}{2}\right)$ depende del cuadrante en el que se encuentra el lado final de $\frac{x}{2}$ y el signo de $\text{RT}\left(\frac{x}{2}\right)$ en ese cuadrante.

\textbf{Identidades auxiliares}

\begin{itemize}
    \item $\tan\left(\dfrac{x}{2}\right) \myequal \csc\left(x\right) - \cot\left(x\right)$
    \item $\cot\left(\dfrac{x}{2}\right) \myequal \csc\left(x\right) + \cot\left(x\right)$
\end{itemize}

\subsection{Identidades trigonométricas del ángulo triple}

\begin{itemize}
    \item $\sin\left(3x\right) \myequal 3 \sin\left(x\right) - 4 \sin^3\left(x\right)$
    \item $\cos\left(3x\right) \myequal 4 \cos^3\left(x\right) - 3 \cos\left(x\right)$
    \item $\tan\left(3x\right) \myequal \dfrac{3 \tan\left(x\right) - \tan^3\left(x\right)}{1 - 3 \tan^2\left(x\right)}$
\end{itemize}

\textbf{Identidades de degradación}

\begin{itemize}
    \item $4\sin^3\left(x\right) \myequal 3 \sin\left(x\right) - \sin\left(3x\right)$
    \item $4\cos^3\left(x\right) \myequal 3 \cos\left(x\right) + \cos\left(3x\right)$
\end{itemize}

\textbf{Identidades auxiliares}

\begin{itemize}
    \item $\dfrac{\sin\left(3x\right)}{\sin\left(x\right)} \myequal 2 \cos\left(2x\right) + 1$
    \item $\dfrac{\cos\left(3x\right)}{\cos\left(x\right)} \myequal 2 \cos\left(2x\right) - 1$
    \item $\dfrac{\tan\left(3x\right)}{\tan\left(x\right)} \myequal \dfrac{2 \cos\left(2x\right) + 1}{2 \cos\left(2x\right) - 1}$
    \item $\sin\left(3x\right) \myequal 4 \sin\left(x\right) \sin\left(60\degree - x\right) \sin\left(60 \degree + x\right)$
    \item $\cos\left(3x\right) \myequal 4 \cos\left(x\right) \cos\left(60\degree - x\right) \cos\left(60 \degree + x\right)$
    \item $\tan\left(3x\right) \myequal 4 \tan\left(x\right) \tan\left(60\degree - x\right) \tan\left(60 \degree + x\right)$
\end{itemize}

\section{Identidades de transformaciones trigonométricas}

\subsection{Identidades de transformación de suma o diferencia de senos o cosenos a producto}

Transformación de suma de senos a producto.

\begin{itemize}
  \item $\sin\left(\alpha\right) + \sin\left(\beta\right) \myequal 2 \sin\left(\dfrac{\alpha + \beta}{2}\right) \cos\left(\dfrac{\alpha - \beta}{2}\right)$
\end{itemize}

Transformación de diferencia de senos a producto.

\begin{itemize}
  \item $\sin\left(\alpha\right) - \sin\left(\beta\right) \myequal 2 \cos\left(\dfrac{\alpha + \beta}{2}\right) \sin\left(\dfrac{\alpha - \beta}{2}\right)$
\end{itemize}

Transformación de suma de cosenos a producto.

\begin{itemize}
  \item $\cos\left(\alpha\right) + \cos\left(\beta\right) \myequal 2 \cos\left(\dfrac{\alpha + \beta}{2}\right) \cos\left(\dfrac{\alpha - \beta}{2}\right)$
\end{itemize}

Transformación de diferencia de cosenos a producto.

\begin{itemize}
  \item $\cos\left(\alpha\right) - \cos\left(\beta\right) \myequal - 2 \sin\left(\dfrac{\alpha + \beta}{2}\right) \sin\left(\dfrac{\alpha - \beta}{2}\right)$
\end{itemize}

\textbf{Identidades auxiliares}

\begin{itemize}
  \item $\sin\left(x\right) + \sin\left(y\right) + \sin\left(z\right) - \sin\left(x + y + z\right) \\ \myequal 4 \sin\left(\dfrac{x+y}{2}\right) \sin\left(\dfrac{y+z}{2}\right) \sin\left(\dfrac{x+z}{2}\right)$
  \item $\cos\left(x\right) + \cos\left(y\right) + \cos\left(z\right) + \cos\left(x + y + z\right) \\ \myequal 4 \cos\left(\dfrac{x+y}{2}\right) \cos\left(\dfrac{y+z}{2}\right) \cos\left(\dfrac{x+z}{2}\right)$
\end{itemize}

\textbf{Identidades auxiliares condicionales}

Si $A + B + C = 180\degree$, entonces

\begin{itemize}
  \item $\sin\left(A\right) + \sin\left(B\right) + \sin\left(C\right) \\ \myequal 4 \cos\left(\dfrac{A}{2}\right) \cos\left(\dfrac{B}{2}\right) \cos\left(\dfrac{C}{2}\right)$
  \item $\cos\left(A\right) + \cos\left(B\right) + \cos\left(C\right) \\ \myequal 4 \sin\left(\dfrac{A}{2}\right) \sin\left(\dfrac{B}{2}\right) \sin\left(\dfrac{C}{2}\right) + 1$
  \item $\sin\left(2A\right) + \sin\left(2B\right) + \sin\left(2C\right) \\ \myequal 4 \sin\left(A\right) \sin\left(B\right) \sin\left(C\right)$
  \item $\cos\left(2A\right) + \cos\left(2B\right) + \cos\left(2C\right) \\ \myequal -4 \cos\left(A\right) \cos\left(B\right) \cos\left(C\right) - 1$
\end{itemize}

\subsection{Identidades de transformación de un producto de senos y/o cosenos a suma o diferencia}

Transformación de producto de senos a suma.

\begin{itemize}
  \item $2 \sin\left(x\right) \cos\left(y\right) \myequal \sin\left(x+y\right) + \sin\left(x-y\right)$
\end{itemize}

Transformación de producto de cosenos a suma.

\begin{itemize}
  \item $2 \cos\left(x\right) \cos\left(y\right) \myequal \cos\left(x+y\right) + \cos\left(x-y\right)$
\end{itemize}

Transformación de producto de senos a diferencia de cosenos.

\begin{itemize}
  \item $2 \sin\left(x\right) \sin\left(y\right) \myequal \cos\left(x-y\right) - \cos\left(x + y\right)$
\end{itemize}

\subsection{Identidades de transformación de cociente a suma o diferencia de cosenos}

Identidades de transformación de cociente de ángulos múltiples del seno.

\begin{itemize}
  \item $\dfrac{\sin\left(3x\right)}{\sin\left(x\right)} \myequal 2 \cos\left(2x\right) + 1$
  \item $\dfrac{\sin\left(5x\right)}{\sin\left(x\right)} \myequal 2 \cos\left(4x\right) + 2 \cos\left(2x\right) + 1$
  \item $\dfrac{\sin\left(7x\right)}{\sin\left(x\right)} \myequal 2 \cos\left(6x\right) + 2 \cos\left(4x\right) + 2 \cos\left(2x\right) + 1$
\end{itemize}

Identidades de transformación de cociente de ángulos múltiples del coseno.

\begin{itemize}
  \item $\dfrac{\cos\left(3x\right)}{\cos\left(x\right)} \myequal 2 \cos\left(2x\right) - 1$
  \item $\dfrac{\cos\left(5x\right)}{\cos\left(x\right)} \myequal 2 \cos\left(4x\right) - 2 \cos\left(2x\right) + 1$
  \item $\dfrac{\cos\left(7x\right)}{\cos\left(x\right)} \myequal 2 \cos\left(6x\right) - 2 \cos\left(4x\right) + 2 \cos\left(2x\right) - 1$
\end{itemize}

\section{Series, sumatorias y productorias trigonométricas}

\subsection{Suma de senos o cosenos de argumentos que están en progresión aritmética}

\begin{itemize}
  \item $\displaystyle \sum_{k=1}^{n} \sin\left(x + kr\right) \myequal \dfrac{\sin\left(\dfrac{nr}{2}\right)}{\sin\left(\dfrac{r}{2}\right)} \cdot \sin\left(\dfrac{P+U}{2}\right)$
  \item $\displaystyle \sum_{k=1}^{n} \cos\left(x + kr\right) \myequal \dfrac{\sin\left(\dfrac{nr}{2}\right)}{\sin\left(\dfrac{r}{2}\right)} \cdot \cos\left(\dfrac{P+U}{2}\right)$
\end{itemize}

donde:

\begin{itemize}
  \item $n$: número de términos
  \item $r$: razón de la progresión aritmética
  \item $P$: primer argumento de la sumatoria
  \item $U$: último argumento de la sumatoria
\end{itemize}

Teoremas adicionales

\begin{itemize}
  \item $\displaystyle \sum_{k=1}^{n} \cos\left(\left(\dfrac{2k}{2n+1}\right)\pi\right) \myequal - \dfrac{1}{2}$
  \item $\displaystyle \sum_{k=1}^{n} \cos\left(\left(\dfrac{2k-1}{2n+1}\right)\pi\right) \myequal \dfrac{1}{2}$
\end{itemize}

\subsection{Productos de R.T. de arcos de la forma $\left(\dfrac{k \pi}{2n+1}\right), n, k \in \mathbb{N}$}

\begin{itemize}
  \item $\displaystyle \prod_{k=1}^{n} \sin \left(\dfrac{k \pi}{2n + 1}\right) \myequal \dfrac{\sqrt{2n+1}}{2^n}$
  \item $\displaystyle \prod_{k=1}^{n} \cos \left(\dfrac{k \pi}{2n + 1}\right) \myequal \dfrac{1}{2^n}$
  \item $\displaystyle \prod_{k=1}^{n} \tan \left(\dfrac{k \pi}{2n + 1}\right) \myequal \sqrt{2n+1}$
\end{itemize}

\end{multicols}
\end{document}
