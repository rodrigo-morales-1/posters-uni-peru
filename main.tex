\documentclass[twocolumn]{article}

% The package amsmath defines the environment equation*
\usepackage{amsmath}
% The package amssymb adds the macro mathbb
\usepackage{amssymb}
\usepackage{parskip}
% The package geometry allows to reduce the margin
\usepackage[
    showframe,
    top=0.5cm,
    left=0.5cm,
    right=0.5cm,
    bottom=1.5cm
]{geometry}

% The variables columnsep sets the horizontal padding for the vertical
% separator between columns.
\setlength{\columnsep}{0.2cm}
% The variable columnseprule sets the width for the vertical separator
% between columns.
\setlength{\columnseprule}{0.5pt}

\usepackage{titlesec}
\titleformat{\section}  % which section command to format
  {\fontsize{8}{10}\bfseries} % format for whole line
  {\thesection} % how to show number
  {1em} % space between number and text
  {} % formatting for just the text
  [] % formatting for after the text

\titleformat{\subsection}
  {\fontsize{8}{10}\bfseries} % format for whole line
  {\thesubsection} % how to show number
  {1em} % space between number and text
  {} % formatting for just the text
  [] % formatting for after the text

\begin{document}
\scriptsize

\section{Identidades trigonométricas de ángulos simples}

Identidades trigonométricas de recíprocas

\begin{equation*}
    \begin{aligned}
        \sin\left(x\right) \cdot \csc\left(x\right) &= 1 \\
        \cos\left(x\right) \cdot \sec\left(x\right) &= 1 \\
        \tan\left(x\right) \cdot \cot\left(x\right) &= 1
    \end{aligned}
\end{equation*}

Identidades trigonométricas por división

\begin{equation*}
    \begin{aligned}
        \tan\left(x\right) &= \frac{\sin\left(x\right)}{\cos\left(x\right)} \\
        \cot\left(x\right) &= \frac{\cos\left(x\right)}{\sin\left(x\right)}
    \end{aligned}
\end{equation*}

Identidades trigonometricas pitagóricas

\begin{equation*}
    \begin{aligned}
        \sin^2\left(x\right) + \cos^2\left(x\right) &= 1 \\
        \sec^2\left(x\right) + \tan^2\left(x\right) &= 1 \\
        \csc^2\left(x\right) + \cot^2\left(x\right) &= 1
    \end{aligned}
\end{equation*}

Identidades trigonométricas auxiliares

\begin{equation*}
    \begin{aligned}
        \left( \sin\left(x\right) \pm \cos\left(x\right) \right)^2 &= 1 \pm 2 \sin\left(x\right) \cos\left(x\right) \\
        \sin^4\left(x\right) + \cos^4\left(x\right) &= 1 - 2 \sin^2\left(x\right) \cos^2\left(x\right) \\
        \sin^6\left(x\right) + \cos^6\left(x\right) &= 1 - 3 \sin^2\left(x\right) \cos^2\left(x\right) \\
        \tan\left(x\right) + \cot\left(x\right) &= \sec\left(x\right) \csc\left(x\right) \\
        \sec^2 \left(x\right) + \csc^2 \left(x\right) &= \sec^2\left(x\right) \csc^2 \left(x\right)
    \end{aligned}
\end{equation*}

\begin{equation*}
    \begin{aligned}
        \sec\left(x\right) + \tan\left(x\right) = m &\rightarrow \sec\left(x\right) - \tan\left(x\right) = \frac{1}{m} \\
        \csc\left(x\right) + \cot\left(x\right) = m &\rightarrow \csc\left(x\right) - \cot\left(x\right) = \frac{1}{m}
    \end{aligned}
\end{equation*}

\begin{equation*}
  \left( 1 + \pm \sin\left(x\right) \pm \cos\left(x\right) \right)^2 = 2 \left(1 \pm \sin\left(x\right)\right) \left(1 \pm \cos\left(x\right)\right)
\end{equation*}

\begin{equation*}
  a \cdot \sin\left(x\right) + b \cos\left(x\right) = c \land c = \sqrt{a^2 + b^2} \rightarrow \sin\left(x\right) = \frac{a}{c} \land \cos\left(x\right) = \frac{b}{c}
\end{equation*}

\section{Identidades trigonométricas de ángulos compuestos}

Identidades trigonométricas de la suma y diferencia de dos ángulos

\begin{equation*}
    \begin{aligned}
        \sin \left(x \pm y\right) &= \sin\left(x\right) \cos\left(y\right) \pm \cos\left(x\right) \sin\left(y\right) \\
        \cos \left(x \pm y\right) &= \cos\left(x\right) \cos\left(y\right) \mp \sin\left(x\right) \sin\left(y\right) \\
        \tan \left(x \pm y\right) &= \frac{\tan\left(x\right)\pm \tan\left(y\right)}{1 \mp \tan\left(x\right) \tan\left(y\right)}
    \end{aligned}
\end{equation*}

Identidades trigonométricas auxiliares para la suma y diferencia de dos ángulos

\begin{equation*}
    \begin{aligned}
        \sin\left(x + y\right) \cdot \sin\left(x - y\right) &= \sin^2\left(x\right) - \sin^2\left(y\right) \\
        \cos\left(x + y\right) \cdot \cos\left(x - y\right) &= \cos^2\left(x\right) - \sin^2\left(y\right)
    \end{aligned}
\end{equation*}

\begin{equation*}
  \tan\left(\theta\right) = \frac{b}{a} \leftrightarrow a \cdot \sin\left(x\right) \pm b \cdot \cos\left(x\right) = \sqrt{a^2 + b^2} \cdot \sin\left(x \pm \theta\right)
\end{equation*}

\begin{equation*}
  \forall x \in \mathbb{R} : - \sqrt{a^2 + b^2} \leq a \cdot \sin\left(x\right) + b \cdot \cos\left(x\right) \leq \sqrt{a^2 + b^2}
\end{equation*}

\begin{equation*}
  \tan\left(x \pm y\right) = \tan\left(x\right) \pm \tan\left(y\right) \pm \tan\left(x\right) \tan\left(y\right) \tan\left(x \pm y\right)
\end{equation*}

Identidades trigonométricas para tres ángulos

\begin{alignat*}{2}
  x + y + z = k \pi \land k \in \mathbb{Z} \rightarrow & \tan\left(x\right) + \tan\left(y\right) + \tan\left(z\right) = \tan\left(x\right) \tan\left(y\right) \tan\left(z\right)
  \\ & \land \cot\left(x\right) \cot\left(y\right) + \cot\left(y\right) \cot\left(z\right) + \cot\left(z\right) \cot\left(x\right) = 1
\end{alignat*}

\begin{alignat*}{2}
  x + y + z = \frac{\left(2k+1\right) \pi}{2} \land k \in \mathbb{Z} \rightarrow & \cot\left(x\right) + \cot\left(y\right) + \cot\left(z\right) = \cot\left(x\right) \cot\left(y\right) \cot\left(z\right)
  \\ & \land \tan\left(x\right) \tan\left(y\right) + \tan\left(y\right) \tan\left(z\right) + \tan\left(z\right) \tan\left(x\right) = 1
\end{alignat*}

\section{Identidades trigonométricas de múltiples ángulos}

\subsection{Identidades trigonométricas del ángulo doble}

Identidades trigonométricas del ángulo doble (parte 1)

\begin{equation*}
    \begin{aligned}
        \sin\left(2x\right) &= 2 \sin\left(x\right) \cos\left(x\right) \\
        \cos\left(2x\right) &= \cos^2\left(x\right) - \sin^2\left(x\right) \\
        \tan\left(2x\right) &= \frac{2 \tan\left(x\right)}{1 - \tan^2\left(x\right)}
    \end{aligned}
\end{equation*}

Identidades de degradación

\begin{equation*}
    \begin{aligned}
        2 \sin^2\left(x\right) &= 1 - \cos\left(2x\right) \\
        2 \cos^2\left(x\right) &= 1 + \cos\left(2x\right)
    \end{aligned}
\end{equation*}

Identidades trigonométricas del ángulo doble (parte 2)

\begin{equation*}
    \begin{aligned}
        \sin\left(2x\right) &= \frac{2 \tan\left(x\right)}{1 + \tan^2\left(x\right)} \\
        \cos\left(2x\right) &= \frac{1 - \tan^2\left(x\right)}{1 + \tan^2\left(x\right)}
    \end{aligned}
\end{equation*}

Identidades auxiliares

\begin{equation*}
    \begin{aligned}
        \left(\sin\left(x\right) \pm \cos\left(x\right)\right)^2 = 1 \pm \sin\left(2x\right) \\
        \sqrt{1 \pm \sin\left(2x\right)} = \left|\sin\left(x\right) \pm \cos\left(x\right)\right| \\
        \cot\left(x\right) + \tan\left(x\right) = 2 \csc\left(2x\right) \\
        \cot\left(x\right) - \tan\left(x\right) = 2 \cot\left(2x\right)
    \end{aligned}
\end{equation*}

Otras fórmulas de degradación

\begin{equation*}
    \begin{aligned}
        8 \sin^4\left(x\right) &= 3 - 4 \cos\left(2x\right) + \cos\left(4x\right) \\
        8 \cos^4\left(x\right) &= 3 + 4 \cos\left(2x\right) + \cos\left(4x\right) \\
        \sin^4\left(x\right) + \cos^4\left(x\right) &= \frac{3}{4} + \frac{1}{4} \cos\left(4x\right) \\
        \sin^6\left(x\right) + \cos^6\left(x\right) &= \frac{5}{8} + \frac{3}{8} \cos\left(4x\right)
    \end{aligned}
\end{equation*}

\subsection{Identidades trigonométricas del ángulo mitad}

\begin{equation*}
    \begin{aligned}
        \sin\left(\frac{x}{2}\right) &= \pm \sqrt{\frac{1 - \cos\left(x\right)}{2}} \\
        \cos\left(\frac{x}{2}\right) &= \pm \sqrt{\frac{1 + \cos\left(x\right)}{2}} \\
        \tan\left(\frac{x}{2}\right) &= \pm \sqrt{\frac{1 - \cos\left(x\right)}{1 + \cos\left(x\right)}}
    \end{aligned}
\end{equation*}

Identidades auxiliares del ángulo mitad

\begin{equation*}
    \begin{aligned}
        \tan\left(\frac{x}{2}\right) &= \csc\left(x\right) - \cot\left(x\right) \\
        \cot\left(\frac{x}{2}\right) &= \csc\left(x\right) + \cot\left(x\right)
    \end{aligned}
\end{equation*}

\subsection{Identidades trigonométricas del ángulo triple}

\begin{equation*}
    \begin{aligned}
        \sin\left(3x\right) &= 3 \sin\left(x\right) - 4 \sin^3\left(x\right) \\
        \cos\left(3x\right) &= 4 \cos^3\left(x\right) - 3 \cos\left(x\right) \\
        \tan\left(3x\right) &= \frac{3 \tan\left(x\right) - \tan^3\left(x\right)}{1 - 3 \tan^2\left(x\right)}
    \end{aligned}
\end{equation*}

Identidades de degradación

\begin{equation*}
    \begin{aligned}
        4\sin^3\left(x\right) &= 3 \sin\left(x\right) - \sin\left(3x\right) \\
        4\cos^3\left(x\right) &= 3 \cos\left(x\right) + \cos\left(3x\right)
    \end{aligned}
\end{equation*}


\end{document}
