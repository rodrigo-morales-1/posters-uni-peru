\documentclass[twocolumn]{article}

% The package amsmath defines the environment equation*
\usepackage{amsmath}
% The package amssymb adds the macro mathbb
\usepackage{amssymb}
\usepackage{parskip}
\usepackage[showframe,top=0.5cm, left=0.5cm, right=0.5cm, bottom=1.5cm]{geometry}


\usepackage{titlesec}
\titleformat{\section}  % which section command to format
  {\fontsize{8}{10}\bfseries} % format for whole line
  {\thesection} % how to show number
  {1em} % space between number and text
  {} % formatting for just the text
  [] % formatting for after the text

\begin{document}
\scriptsize

Identidades trigonométricas de recíprocas

\begin{equation*}
  \sin\left(x\right) \cdot \csc\left(x\right) = 1
\end{equation*}

\begin{equation*}
  \cos\left(x\right) \cdot \sec\left(x\right) = 1
\end{equation*}

\begin{equation*}
  \tan\left(x\right) \cdot \cot\left(x\right) = 1
\end{equation*}

Identidades trigonométricas por división

\begin{equation*}
  \tan\left(x\right) = \frac{\sin\left(x\right)}{\cos\left(x\right)}
\end{equation*}

\begin{equation*}
  \cot\left(x\right) = \frac{\cos\left(x\right)}{\sin\left(x\right)}
\end{equation*}

Identidades trigonometricas pitagóricas

\begin{equation*}
  \sin^2\left(x\right) + \cos^2\left(x\right) = 1
\end{equation*}

\begin{equation*}
  \sec^2\left(x\right) + \tan^2\left(x\right) = 1
\end{equation*}

\begin{equation*}
  \csc^2\left(x\right) + \cot^2\left(x\right) = 1
\end{equation*}

Identidades trigonométricas auxiliares

\begin{equation*}
  \left( \sin\left(x\right) + \cos\left(x\right) \right)^2 = 1 + 2 \sin\left(x\right) \cos\left(x\right)
\end{equation*}

\begin{equation*}
  \left( \sin\left(x\right) - \cos\left(x\right) \right)^2 = 1 - 2 \sin\left(x\right) \cos\left(x\right)
\end{equation*}

\begin{equation*}
  \sin^4\left(x\right) + \cos^4\left(x\right) = 1 - 2 \sin^2\left(x\right) \cos^2\left(x\right)
\end{equation*}

\begin{equation*}
  \sin^6\left(x\right) + \cos^6\left(x\right) = 1 - 3 \sin^2\left(x\right) \cos^2\left(x\right)
\end{equation*}

\begin{equation*}
  \tan\left(x\right) + \cot\left(x\right) = \sec\left(x\right) \csc\left(x\right)
\end{equation*}

\begin{equation*}
  \sec^2 \left(x\right) + \csc^2 \left(x\right) = \sec^2\left(x\right) \csc^2 \left(x\right)
\end{equation*}

\begin{equation*}
  \sec\left(x\right) + \tan\left(x\right) = m \rightarrow \sec\left(x\right) - \tan\left(x\right) = \frac{1}{m}
\end{equation*}

\begin{equation*}
  \csc\left(x\right) + \cot\left(x\right) = m \rightarrow \csc\left(x\right) - \cot\left(x\right) = \frac{1}{m}
\end{equation*}

\begin{equation*}
  \left( 1 + \pm \sin\left(x\right) \pm \cos\left(x\right) \right)^2 = 2 \left(1 \pm \sin\left(x\right)\right) \left(1 \pm \cos\left(x\right)\right)
\end{equation*}

\begin{equation*}
  a \cdot \sin\left(x\right) + b \cos\left(x\right) = c \land c = \sqrt{a^2 + b^2} \rightarrow \sin\left(x\right) = \frac{a}{c} \land \cos\left(x\right) = \frac{b}{c}
\end{equation*}

Identidades trigonométricas de la suma y diferencia de dos arcos

\begin{equation*}
  \sin \left(x \pm y\right) = \sin\left(x\right) \cos\left(y\right) \pm \cos\left(x\right) \sin\left(y\right)
\end{equation*}

\begin{equation*}
  \cos \left(x \pm y\right) = \cos\left(x\right) \cos\left(y\right) \mp \sin\left(x\right) \sin\left(y\right)
\end{equation*}

\begin{equation*}
  \tan \left(x \pm y\right) = \frac{\tan\left(x\right)\pm \tan\left(y\right)}{1 \mp \tan\left(x\right) \tan\left(y\right)}
\end{equation*}

Identidades trigonométricas auxiliares para la suma y diferencia de dos arcos

\begin{equation*}
  \sin\left(x + y\right) \cdot \sin\left(x - y\right) = \sin^2\left(x\right) - \sin^2\left(y\right)
\end{equation*}

\begin{equation*}
  \cos\left(x + y\right) \cdot \cos\left(x - y\right) = \cos^2\left(x\right) - \sin^2\left(y\right)
\end{equation*}

\begin{equation*}
  \tan\left(\theta\right) = \frac{b}{a} \leftrightarrow a \cdot \sin\left(x\right) \pm b \cdot \cos\left(x\right) = \sqrt{a^2 + b^2} \cdot \sin\left(x \pm \theta\right)
\end{equation*}

\begin{equation*}
  \forall x \in \mathbb{R} : - \sqrt{a^2 + b^2} \leq a \cdot \sin\left(x\right) + b \cdot \cos\left(x\right) \leq \sqrt{a^2 + b^2}
\end{equation*}

\begin{equation*}
  \tan\left(x \pm y\right) = \tan\left(x\right) \pm \tan\left(y\right) \pm \tan\left(x\right) \tan\left(y\right) \tan\left(x \pm y\right)
\end{equation*}

Identidades trigonométricas para tres arcos

\begin{alignat*}{2}
  x + y + z = k \pi \land k \in \mathbb{Z} \rightarrow & \tan\left(x\right) + \tan\left(y\right) + \tan\left(z\right) = \tan\left(x\right) \tan\left(y\right) \tan\left(z\right)
  \\ & \land \cot\left(x\right) \cot\left(y\right) + \cot\left(y\right) \cot\left(z\right) + \cot\left(z\right) \cot\left(x\right) = 1
\end{alignat*}

\begin{alignat*}{2}
  x + y + z = \frac{\left(2k+1\right) \pi}{2} \land k \in \mathbb{Z} \rightarrow & \cot\left(x\right) + \cot\left(y\right) + \cot\left(z\right) = \cot\left(x\right) \cot\left(y\right) \cot\left(z\right)
  \\ & \land \tan\left(x\right) \tan\left(y\right) + \tan\left(y\right) \tan\left(z\right) + \tan\left(z\right) \tan\left(x\right) = 1
\end{alignat*}

Identidades trigonométricas del arco doble (parte 1)

\begin{equation*}
  \sin\left(2x\right) = 2 \sin\left(x\right) \cos\left(x\right)
\end{equation*}

\begin{equation*}
  \cos\left(2x\right) = \cos^2\left(x\right) - \sin^2\left(x\right)
\end{equation*}

\begin{equation*}
  \tan\left(2x\right) = \frac{2 \tan\left(x\right)}{1 - \tan^2\left(x\right)}
\end{equation*}

Identidades de degradación

\begin{equation*}
  2 \sin^2\left(x\right) = 1 - \cos\left(2x\right)
\end{equation*}

\begin{equation*}
  2 \cos^2\left(x\right) = 1 + \cos\left(2x\right)
\end{equation*}

Identidades trigonométricas del arco doble (parte 2)

\begin{equation*}
  \sin\left(2x\right) = \frac{2 \tan\left(x\right)}{1 + \tan^2\left(x\right)}
\end{equation*}

\begin{equation*}
  \cos\left(2x\right) = \frac{1 - \tan^2\left(x\right)}{1 + \tan^2\left(x\right)}
\end{equation*}

Identidades auxiliares

\begin{equation*}
  \left(\sin\left(x\right) \pm \cos\left(x\right)\right)^2 = 1 \pm \sin\left(2x\right)
\end{equation*}

\begin{equation*}
  \sqrt{1 \pm \sin\left(2x\right)} = \left|\sin\left(x\right) \pm \cos\left(x\right)\right|
\end{equation*}

\begin{equation*}
  \cot\left(x\right) + \tan\left(x\right) = 2 \csc\left(2x\right)
\end{equation*}

\begin{equation*}
  \cot\left(x\right) - \tan\left(x\right) = 2 \cot\left(2x\right)
\end{equation*}

Otras fórmulas de degradación

\begin{equation*}
  8 \sin^4\left(x\right) = 3 - 4 \cos\left(2x\right) + \cos\left(4x\right)
\end{equation*}

\begin{equation*}
  8 \cos^4\left(x\right) = 3 + 4 \cos\left(2x\right) + \cos\left(4x\right)
\end{equation*}

\begin{equation*}
  \sin^4\left(x\right) + \cos^4\left(x\right) = \frac{3}{4} + \frac{1}{4} \cos\left(4x\right)
\end{equation*}

\begin{equation*}
  \sin^6\left(x\right) + \cos^6\left(x\right) = \frac{5}{8} + \frac{3}{8} \cos\left(4x\right)
\end{equation*}

Identidades trigonometricas del arco mitad

\begin{equation*}
  \sin\left(\frac{x}{2}\right) = \pm \sqrt{\frac{1 - \cos\left(x\right)}{2}}
\end{equation*}

\begin{equation*}
  \cos\left(\frac{x}{2}\right) = \pm \sqrt{\frac{1 + \cos\left(x\right)}{2}}
\end{equation*}

\begin{equation*}
  \tan\left(\frac{x}{2}\right) = \pm \sqrt{\frac{1 - \cos\left(x\right)}{1 + \cos\left(x\right)}}
\end{equation*}

Identidades auxiliares del arco mitad

\begin{equation*}
  \tan\left(\frac{x}{2}\right) = \csc\left(x\right) - \cot\left(x\right)
\end{equation*}

\begin{equation*}
  \cot\left(\frac{x}{2}\right) = \csc\left(x\right) + \cot\left(x\right)
\end{equation*}

Identidades trigonométricsa del arco triple

\begin{equation*}
  \sin\left(3x\right) = 3 \sin\left(x\right) - 4 \sin^3\left(x\right)
\end{equation*}

\begin{equation*}
  \cos\left(3x\right) = 4 \cos^3\left(x\right) - 3 \cos\left(x\right)
\end{equation*}

\begin{equation*}
  \tan\left(3x\right) = \frac{3 \tan\left(x\right) - \tan^3\left(x\right)}{1 - 3 \tan^2\left(x\right)}
\end{equation*}

Identidades de degradación

\begin{equation*}
  4\sin^3\left(x\right) = 3 \sin\left(x\right) - \sin\left(3x\right)
\end{equation*}

\begin{equation*}
  4\cos^3\left(x\right) = 3 \cos\left(x\right) + \cos\left(3x\right)
\end{equation*}


\end{document}


