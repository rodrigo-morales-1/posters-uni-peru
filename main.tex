\documentclass{article}

\usepackage{amsmath}

\begin{document}

Identidades trigonométricas de recíprocas

\begin{equation*}
  \sin\left(x\right) \cdot \csc\left(x\right) = 1
\end{equation*}

\begin{equation*}
  \cos\left(x\right) \cdot \sec\left(x\right) = 1
\end{equation*}

\begin{equation*}
  \tan\left(x\right) \cdot \cot\left(x\right) = 1
\end{equation*}

Identidades trigonométricas por división

\begin{equation*}
  \tan\left(x\right) = \frac{\sin\left(x\right)}{\cos\left(x\right)}
\end{equation*}

\begin{equation*}
  \cot\left(x\right) = \frac{\cos\left(x\right)}{\sin\left(x\right)}
\end{equation*}

Identidades trigonometricas pitagóricas

\begin{equation*}
  \sin^2\left(x\right) + \cos^2\left(x\right) = 1
\end{equation*}

\begin{equation*}
  \sec^2\left(x\right) + \tan^2\left(x\right) = 1
\end{equation*}

\begin{equation*}
  \csc^2\left(x\right) + \cot^2\left(x\right) = 1
\end{equation*}

Identidades trigonométricas auxiliares

\begin{equation*}
  \left( \sin\left(x\right) + \cos\left(x\right) \right)^2 = 1 + 2 \sin\left(x\right) \cos\left(x\right)
\end{equation*}

\begin{equation*}
  \left( \sin\left(x\right) - \cos\left(x\right) \right)^2 = 1 - 2 \sin\left(x\right) \cos\left(x\right)
\end{equation*}

\begin{equation*}
  \sin^4\left(x\right) + \cos^4\left(x\right) = 1 - 2 \sin^2\left(x\right) \cos^2\left(x\right)
\end{equation*}

\begin{equation*}
  \sin^6\left(x\right) + \cos^6\left(x\right) = 1 - 3 \sin^2\left(x\right) \cos^2\left(x\right)
\end{equation*}

\begin{equation*}
  \tan\left(x\right) + \cot\left(x\right) = \sec\left(x\right) \csc\left(x\right)
\end{equation*}

\begin{equation*}
  \sec^2 \left(x\right) + \csc^2 \left(x\right) = \sec^2\left(x\right) \csc^2 \left(x\right)
\end{equation*}

\begin{equation*}
  \sec\left(x\right) + \tan\left(x\right) = m \Rightarrow \sec\left(x\right) - \tan\left(x\right) = \frac{1}{m}
\end{equation*}

\begin{equation*}
  \csc\left(x\right) + \cot\left(x\right) = m \Rightarrow \csc\left(x\right) - \cot\left(x\right) = \frac{1}{m}
\end{equation*}

\begin{equation*}
  \left( 1 + \pm \sin\left(x\right) \pm \cos\left(x\right) \right)^2 = 2 \left(1 \pm \sin\left(x\right)\right) \left(1 \pm \cos\left(x\right)\right)
\end{equation*}

\begin{equation*}
  a \cdot \sin\left(x\right) + b \cos\left(x\right) = c \land c = \sqrt{a^2 + b^2} \Rightarrow \sin\left(x\right) = \frac{a}{c} \land \cos\left(x\right) = \frac{b}{c}
\end{equation*}

\end{document}


