\documentclass{article}

\usepackage{xcolor}
\colorlet{FilledSurface}{blue!20}
\colorlet{FilledSurfaceGroupOne}{blue!20}
\colorlet{FilledSurfaceGroupTwo}{red!20}
\colorlet{FilledSurfaceGroupThree}{green!20}
\colorlet{FilledSurfaceGroupFour}{magenta!20}
\colorlet{FormulaBackground}{green!10}
\colorlet{FormulaFrame}{green}

\usepackage{graphicx}
\usepackage{multicolrule}
\columnseprule=0.5pt
\usepackage[
    a2paper,
    top=0.5cm,
    left=0.5cm,
    right=0.5cm,
    bottom=0.5cm
]{geometry}
\usepackage{wrapfig}
\usepackage{parskip}
\usepackage{tcolorbox}
\usepackage{amsmath}

\newtcbox{\myFormula}{
    on line,
    colback=FormulaBackground, % Sets the background color
    colframe=FormulaFrame,     % Sets the border frame color
    boxrule=1pt,             % Defines the thickness of the outer border frame
    arc=2pt,                   % Sets the radius for the rounded corners
    boxsep=2pt,                % Sets internal padding between the frame and the "left/right/top/bottom" area
    left=1pt,                  % Extra horizontal space on the left of the content
    right=1pt,                 % Extra horizontal space on the right of the content
    top=1pt,                   % Extra vertical space above the content
    bottom=1pt                 % Extra vertical space below the content
}

\newcommand\myPolygonArea[1]{\left[#1\right]}

\begin{document}
\begin{multicols*}{3}

    \section{Área de regiones poligonales}

    \begin{minipage}[t]{0.33\linewidth}
        \begin{center}
            \includegraphics[width=\linewidth]{figuras/área-de-región-cuadrada/main.pdf}
            \myFormula{$\myPolygonArea{ABCD} = l^2$}
        \end{center}
    \end{minipage}
    \begin{minipage}[t]{0.33\linewidth}
        \begin{center}
            \includegraphics[width=\linewidth]{figuras/área-de-región-rectangular/main.pdf}
            \myFormula{$\myPolygonArea{ABCD} = a \cdot b$}
        \end{center}
    \end{minipage}
    \begin{minipage}[t]{0.33\linewidth}
        \begin{center}
            \includegraphics[width=\linewidth]{figuras/área-de-región-paralelográmica-1/main.pdf}
            \myFormula{$\myPolygonArea{ABCD} = b \cdot h$}
        \end{center}
    \end{minipage}

    \section{Área de regiones triangulares}

    \begin{minipage}{0.5\linewidth}
        \begin{center}
        \includegraphics[width=\linewidth]{figuras/área-de-región-triangular-1/main.pdf}
        \myFormula{$\myPolygonArea{ABC} = \dfrac{bh}{2}$}
        \end{center}
    \end{minipage}
    \begin{minipage}{0.5\linewidth}
        \begin{center}
        \includegraphics[width=\linewidth]{figuras/área-de-región-triangular-2/main.pdf}
            \myFormula{$\myPolygonArea{ABC} = \dfrac{bc}{2} \sin\left(\alpha\right)$}
        \end{center}
    \end{minipage}
    \begin{minipage}{0.5\linewidth}
        \begin{center}
            \includegraphics[width=\linewidth]{figuras/área-de-región-triangular-3/main.pdf}
            \par $p$: semiperímetro
            \par $p = \dfrac{a+b+c}{2}$
            \par \myFormula{$\myPolygonArea{ABC} = \sqrt{p \left(p-a\right)\left(p-b\right)\left(p-c\right)}$}
        \end{center}
    \end{minipage}
    \begin{minipage}{0.5\linewidth}
        \begin{center}
        \includegraphics[width=\linewidth]{figuras/área-de-región-triangular-en-función-del-inradio/main.pdf}
        \par $p$: semiperímetro
        \par $p = \dfrac{a+b+c}{2}$
        \par \myFormula{$\myPolygonArea{ABC} = p \cdot r$}
        \end{center}
    \end{minipage}
    \begin{minipage}{0.5\linewidth}
        \begin{center}
        \includegraphics[width=\linewidth]{figuras/área-de-región-triangular-en-función-del-exradio/main.pdf}
        \par $p$: semiperímetro
        \par $p = \dfrac{a+b+c}{2}$
        \par \myFormula{$\myPolygonArea{ABC} = r_a \left(p-a\right)$}
        \end{center}
    \end{minipage}

    \subsection{Relación de áreas en regiones triangulares}

    \begin{minipage}{\linewidth}
        \begin{center}
            Teorema
            \includegraphics[width=\linewidth]{figuras/relación-de-áreas-en-regiones-triangulares-1/main.pdf}
            Si $BH = NP$, entonces \myFormula{$\dfrac{\myPolygonArea{ABC}}{\myPolygonArea{MNL}} = \dfrac{b}{n}$}
        \end{center}
    \end{minipage}
    \begin{minipage}{0.5\linewidth}
        \begin{center}
            Corolario
            \includegraphics[width=\linewidth]{figuras/relación-de-áreas-en-regiones-triangulares-2/main.pdf}
            Si $\overline{BM}$ es mediana, entonces \myFormula{$\myPolygonArea{ABM} = \myPolygonArea{MBC}$}
        \end{center}
    \end{minipage}
    \begin{minipage}{0.5\linewidth}
        \begin{center}
            Corolario
            \includegraphics[width=\linewidth]{figuras/relación-de-áreas-en-regiones-triangulares-3/main.pdf}
            En un triángulo, al trazar las tres medianas, se determinan seis regiones triangulares equivalentes.
        \end{center}
    \end{minipage}
    \begin{minipage}{\linewidth}
        \begin{center}
            Teorema
            \includegraphics[width=\linewidth]{figuras/relación-de-áreas-en-regiones-triangulares-4/main.pdf}
            Si $\alpha = \theta$, entonces \myFormula{$\dfrac{\myPolygonArea{ABC}}{\myPolygonArea{MNL}} = \dfrac{bc}{nl}$}
        \end{center}
    \end{minipage}
    \begin{minipage}{\linewidth}
        \begin{center}
            Teorema
            \includegraphics[width=\linewidth]{figuras/relación-de-áreas-en-regiones-triangulares-5/main.pdf}
            Si $\alpha + \theta = 180$, entonces:
            \myFormula{$\dfrac{\myPolygonArea{ABC}}{\myPolygonArea{MNL}} = \dfrac{bc}{nl}$}
        \end{center}
    \end{minipage}
    \begin{minipage}{\linewidth}
        \begin{center}
            Teorema
            \includegraphics[width=\linewidth]{figuras/relación-de-áreas-en-regiones-triangulares-6/main.pdf}
            Si $\triangle ABC \sim \triangle MNL$, entonces:
            \myFormula{$\dfrac{\myPolygonArea{ABC}}{\myPolygonArea{MNL}} = \dfrac{a^2}{m^2} = \dfrac{b^2}{n^2} + \dfrac{c^2}{l^2} = \dfrac{H^2}{h^2} = \dfrac{R^2}{r^2} = k^2$}
        \end{center}
    \end{minipage}

    \section{Área de regiones cuadrángulares}

    \begin{minipage}{0.5\linewidth}
        \begin{center}
            Área de cuadrilátero inscrito
        \includegraphics[width=\linewidth]{figuras/área-de-región-cuadrangular-inscrita/main.pdf}
        \myFormula{$\left[ABCD\right] = \sqrt{\left(p - a\right) \left(p - b\right) \left(p - c\right) \left(p - d\right)}$}
        \end{center}
    \end{minipage}
    \begin{minipage}{0.5\linewidth}
        \begin{center}
            Área de cuadrilátero circunscrito
        \includegraphics[width=\linewidth]{figuras/área-de-región-cuadrangular-circunscrita/main.pdf}
            \myFormula{$\left[ABCD\right] = p \cdot r$}
            $p$: semiperímetro

            $p = \dfrac{a+b+c+d}{2}$
        \end{center}
    \end{minipage}

    \rule{\linewidth}{0.4pt}

    \begin{minipage}{0.5\linewidth}
        \begin{center}
            Área de cuadrilátero bicéntrico
            \includegraphics[width=\linewidth]{figuras/área-de-región-cuadrangular-bicéntrica/main.pdf}
            \myFormula{$\left[ABCD\right] = \sqrt{abcd}$}
        \end{center}
    \end{minipage}

    \rule{\linewidth}{0.4pt}

    \begin{minipage}{0.5\linewidth}
        \includegraphics[width=\linewidth]{figuras/área-de-región-cuadrangular-usando-diagonales/main.pdf}
        \begin{center}
            \myFormula{$\left[ABCD\right] = \dfrac{AC \cdot BD}{2} \sin\left(\alpha\right)$}
        \end{center}
    \end{minipage}
    \begin{minipage}{0.5\linewidth}
        \includegraphics[width=\linewidth]{figuras/área-de-región-cuadrangular-de-diagonales-perpendiculares/main.pdf}
        \begin{center}
            \myFormula{$\myPolygonArea{ABCD} = \dfrac{AC \cdot BD}{2}$}
        \end{center}
    \end{minipage}

    \subsection{Relación de áreas en regiones cuadrangulares}

    \begin{minipage}{0.5\linewidth}
        \includegraphics[width=\linewidth]{figuras/relación-de-áreas-en-regiones-cuadrangulares-suma-de-2-áreas/main.pdf}
        \begin{center}
            \myFormula{$S_1 + S_3 = S_2 + S_4$}
        \end{center}
    \end{minipage}
    \begin{minipage}{0.5\linewidth}
        \includegraphics[width=\linewidth]{figuras/relación-de-áreas-en-regiones-cuadrangulares-área-determinada-por-segmentos-puntos-medios/main.pdf}
        \begin{center}
            \myFormula{$\myPolygonArea{MNPQ} = \dfrac{1}{2} \myPolygonArea{ABCD}$}
        \end{center}
    \end{minipage}

    \begin{minipage}{0.5\linewidth}
        \includegraphics[width=\linewidth]{figuras/relación-de-áreas-en-regiones-cuadrangulares-producto-de-2-áreas/main.pdf}
        \begin{center}
            \myFormula{$S_1 \cdot S_3 = S_2 \cdot S_4$}
        \end{center}
    \end{minipage}
    \begin{minipage}{0.5\linewidth}
        \includegraphics[width=\linewidth]{figuras/relación-de-áreas-en-regiones-cuadrangulares-puntos-medios-y-ángulo-entre-diagonales/main.pdf}
        \begin{center}
            \myFormula{$\myPolygonArea{ABCD} = MP \cdot NQ \cdot \sin\left(\alpha\right)$}
        \end{center}
    \end{minipage}

    \begin{minipage}{0.5\linewidth}
        \includegraphics[width=\linewidth]{figuras/relación-de-áreas-en-regiones-cuadrangulares-vertices-a-2-puntos-medios/main.pdf}
        \begin{center}
            \myFormula{$\myPolygonArea{AMCN} = \dfrac{\myPolygonArea{ABCD}}{2}$}
        \end{center}
    \end{minipage}
    \begin{minipage}{0.5\linewidth}
        \includegraphics[width=\linewidth]{figuras/relación-de-áreas-en-regiones-cuadrangulares-vertices-a-4-puntos-medios/main.pdf}
        \begin{center}
            \myFormula{$S_1 + S_3 = S_2 + S_4$}
        \end{center}
    \end{minipage}

    \begin{minipage}{0.5\linewidth}
        \includegraphics[width=\linewidth]{figuras/relación-de-áreas-en-regiones-cuadrangulares-área-de-cuadrilátero-interior-determinado-por-medianas/main.pdf}
        \begin{center}
            \myFormula{$S = S_1 + S_2 + S_3 + S_4$}
        \end{center}
    \end{minipage}
    \begin{minipage}{0.5\linewidth}
        \includegraphics[width=\linewidth]{figuras/relación-de-áreas-en-regiones-cuadrangulares-suma-de-áreas-determinadas-por-segmentos-puntos-medios/main.pdf}
        \begin{center}
            \myFormula{$S_1 + S_3 = S_2 + S_4$}
        \end{center}
    \end{minipage}

    \section{Área de regiones trapeciales}

    \begin{minipage}{0.5\linewidth}
        \includegraphics[width=\linewidth]{figuras/área-de-región-trapecial-usando-altura-y-bases/main.pdf}
        \begin{center}
            \myFormula{$\myPolygonArea{ABCD} = \left(\dfrac{a+b}{2}\right) h$}
        \end{center}
    \end{minipage}
    \begin{minipage}{0.5\linewidth}
        \includegraphics[width=\linewidth]{figuras/área-de-región-trapecial-usando-segmentos-de-puntos-medios-y-altura/main.pdf}
        \begin{center}
            \myFormula{$\myPolygonArea{ABCD} = MN \cdot h$}
        \end{center}
    \end{minipage}

    \subsection{Relación de áreas en regiones trapeciales}

    \begin{minipage}{0.5\linewidth}
        \includegraphics[width=\linewidth]{figuras/relación-de-áreas-en-regiones-trapeciales-1/main.pdf}
        \begin{center}
            \myFormula{$S_1 = S_2$}
        \end{center}
    \end{minipage}
    \begin{minipage}{0.5\linewidth}
        \includegraphics[width=\linewidth]{figuras/relación-de-áreas-en-regiones-trapeciales-2/main.pdf}
        \begin{center}
            \myFormula{$S = \sqrt{S_1 \cdot S_2}$}
        \end{center}
    \end{minipage}
    \begin{minipage}{0.5\linewidth}
        \includegraphics[width=\linewidth]{figuras/relación-de-áreas-en-regiones-trapeciales-3/main.pdf}
        \begin{center}
            \myFormula{$\myPolygonArea{ABCD} = \left(\sqrt{S_1} + \sqrt{S_2}\right)^2$}
        \end{center}
    \end{minipage}
    \begin{minipage}{0.5\linewidth}
        \includegraphics[width=\linewidth]{figuras/relación-de-áreas-en-regiones-trapeciales-4/main.pdf}
        \begin{center}
            \myFormula{$S = S_1 + S_2$}
        \end{center}
    \end{minipage}
    \begin{minipage}{0.5\linewidth}
        \includegraphics[width=\linewidth]{figuras/relación-de-áreas-en-regiones-trapeciales-5/main.pdf}
        \begin{center}
            \myFormula{$\myPolygonArea{ABCD} = CD \cdot MH$}
        \end{center}
    \end{minipage}
    \begin{minipage}{0.5\linewidth}
        \includegraphics[width=\linewidth]{figuras/relación-de-áreas-en-regiones-trapeciales-6/main.pdf}
        \begin{center}
            \myFormula{$S_1 = S_2$}
        \end{center}
    \end{minipage}

    \subsection{Área de regiones paralelográmicas}

    \begin{minipage}{0.5\linewidth}
        \includegraphics[width=\linewidth]{figuras/área-de-región-paralelográmica-1/main.pdf}
        \begin{center}
            \myFormula{$\myPolygonArea{ABCD} = b \cdot h$}
        \end{center}
    \end{minipage}
    \begin{minipage}{0.5\linewidth}
        \includegraphics[width=\linewidth]{figuras/área-de-región-paralelográmica-2/main.pdf}
        \begin{center}
            \myFormula{$\myPolygonArea{ABCD} = m \cdot n$}
        \end{center}
    \end{minipage}
    \begin{minipage}{0.5\linewidth}
        \includegraphics[width=\linewidth]{figuras/área-de-región-paralelográmica-3/main.pdf}
        \begin{center}
            \myFormula{$\myPolygonArea{ABCD} = a \cdot b \cdot \sin\left(\alpha\right)$}
        \end{center}
    \end{minipage}
    \begin{minipage}{0.5\linewidth}
        \includegraphics[width=\linewidth]{figuras/área-de-región-paralelográmica-4/main.pdf}
        \begin{center}
            \myFormula{$\myPolygonArea{ABCD} = \dfrac{AC \cdot BD}{2}$}
        \end{center}
    \end{minipage}

    \subsection{Relación de áreas en regiones paralelográmicas}

    \begin{minipage}{0.5\linewidth}
        \includegraphics[width=\linewidth]{figuras/relación-de-áreas-en-regiones-paralelográmicas-1/main.pdf}
        \begin{center}
            \myFormula{$S_1 = S_2$}
        \end{center}
    \end{minipage}
    \begin{minipage}{0.5\linewidth}
        \includegraphics[width=\linewidth]{figuras/relación-de-áreas-en-regiones-paralelográmicas-2/main.pdf}
        \begin{center}
            \myFormula{$S_1 = S_2 = S_3 = S_4$}
        \end{center}
    \end{minipage}
    \begin{minipage}{0.5\linewidth}
        \includegraphics[width=\linewidth]{figuras/relación-de-áreas-en-regiones-paralelográmicas-3/main.pdf}
        \begin{center}
            \myFormula{$\myPolygonArea{AFD} = \dfrac{\myPolygonArea{ABCD}}{2}$}
        \end{center}
    \end{minipage}
    \begin{minipage}{0.5\linewidth}
        \includegraphics[width=\linewidth]{figuras/relación-de-áreas-en-regiones-paralelográmicas-4/main.pdf}
        Si $\overline{EF} \parallel \overline{AD}$ y $\overline{GH} \parallel \overline{AB}$, entonces:
        \begin{center}
            \myFormula{$S_1 = S_2$}
        \end{center}
    \end{minipage}
    \begin{minipage}{0.5\linewidth}
        \includegraphics[width=\linewidth]{figuras/relación-de-áreas-en-regiones-paralelográmicas-5/main.pdf}
        Si $E \in \overline{AB}$, entonces:
        \begin{center}
            \myFormula{$S = S_1 + S_2$}
        \end{center}
    \end{minipage}
    \begin{minipage}{0.5\linewidth}
        \includegraphics[width=\linewidth]{figuras/relación-de-áreas-en-regiones-paralelográmicas-6/main.pdf}
        Si $E$ está en el interior de $ABCD$, entonces:
        \begin{center}
            \myFormula{$S_1 + S_3 = S_2 + S_4 = \dfrac{\myPolygonArea{ABCD}}{2}$}
        \end{center}
    \end{minipage}

\end{multicols*}
\end{document}


%% Plantilla
%%
%% \begin{minipage}{0.5\linewidth}
%%     \includegraphics[width=\linewidth]{example-image-a}
%%     \begin{center}
%%         \myFormula{$0$}
%%     \end{center}
%% \end{minipage}
%% \begin{minipage}{0.5\linewidth}
%%     \includegraphics[width=\linewidth]{example-image-a}
%%     \begin{center}
%%         \myFormula{$0$}
%%     \end{center}
%% \end{minipage}
