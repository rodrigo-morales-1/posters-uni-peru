\documentclass{article}

\usepackage{xcolor}
\colorlet{FilledSurface}{blue!20}
\colorlet{FilledSurfaceGroupOne}{blue!20}
\colorlet{FilledSurfaceGroupTwo}{red!20}
\colorlet{FilledSurfaceGroupThree}{green!20}
\colorlet{FilledSurfaceGroupFour}{magenta!20}
\colorlet{FormulaBackground}{green!10}
\colorlet{FormulaFrame}{green}

\usepackage{graphicx}
\usepackage{multicolrule}
\columnseprule=0.5pt
\usepackage[
    a2paper,
    top=0.5cm,
    left=0.5cm,
    right=0.5cm,
    bottom=0.5cm
]{geometry}
\usepackage{wrapfig}
\usepackage{parskip}
\usepackage{tcolorbox}
\usepackage{amsmath}

\newtcbox{\myformula}{
    colback=FormulaBackground, % Sets the background color
    colframe=FormulaFrame,     % Sets the border frame color
    boxrule=1pt,             % Defines the thickness of the outer border frame
    arc=2pt,                   % Sets the radius for the rounded corners
    boxsep=2pt,                % Sets internal padding between the frame and the "left/right/top/bottom" area
    left=1pt,                  % Extra horizontal space on the left of the content
    right=1pt,                 % Extra horizontal space on the right of the content
    top=1pt,                   % Extra vertical space above the content
    bottom=1pt                 % Extra vertical space below the content
}

\begin{document}
\begin{multicols*}{3}
    \section{Área de regiones cuadrángulares}

    \subsection{Área de cuadrilátero bicéntrico}

    \begin{minipage}{0.5\linewidth}
        El área de una región cuadrangular determinado por un cuadrilátero inscrito a una circunferencia y circunscrito a otra circunferencia es igual a la raíz cuadrada del producto de las longitudes de los lados.

        \myformula{$\left[ABCD\right] = \sqrt{abcd}$}
    \end{minipage}
    \begin{minipage}{0.5\linewidth}
        \includegraphics[width=\linewidth]{figuras/área-de-cuadrilátero-bicéntrico/main.pdf}
    \end{minipage}

    \subsection{Área de cuadrilátero circunscrito}

    \begin{minipage}{0.5\linewidth}
        El área de una región cuadrangular determinado por un cuadrilátero circunscrito a una circunferencia es igual al producto del semiperímetro del cuadrilátero y el radio de la circunferencia inscrita.

        Semiperímetro: $p = \dfrac{a+b+c+d}{2}$

        \myformula{$\left[ABCD\right] = p \cdot r$}

    \end{minipage}
    \begin{minipage}{0.5\linewidth}
        \includegraphics[width=\linewidth]{figuras/área-de-cuadrilátero-circunscrito/main.pdf}
    \end{minipage}

    \subsection{Área de cuadrilátero inscrito}

    \begin{minipage}{0.6\linewidth}
        \myformula{$\left[ABCD\right] = \sqrt{\left(p - a\right) \left(p - b\right) \left(p - c\right) \left(p - d\right)}$}
    \end{minipage}
    \begin{minipage}{0.4\linewidth}
        \includegraphics[width=\linewidth]{figuras/área-de-cuadrilátero-inscrito/main.pdf}
    \end{minipage}

    \subsection{Área usando producto de diagonales y ángulo entre diagonales}

    \begin{minipage}{0.6\linewidth}
        \myformula{$\left[ABCD\right] = \dfrac{AC \cdot BD}{2} \sin\left(\alpha\right)$}
    \end{minipage}
    \begin{minipage}{0.4\linewidth}
        \includegraphics[width=\linewidth]{figuras/área-de-cuadrilátero-usando-diagonales/main.pdf}
    \end{minipage}

\end{multicols*}
\end{document}
