\documentclass{article}

\usepackage{graphicx}
\usepackage{multicolrule}
\columnseprule=0.5pt
\usepackage[
    a2paper,
    top=0.5cm,
    left=0.5cm,
    right=0.5cm,
    bottom=0.5cm
]{geometry}
\usepackage{wrapfig}
\usepackage{parskip}

\begin{document}
\begin{multicols*}{3}
    \section{Área de regiones cuadrángulares}

    \subsection{Área de cuadrilátero bicéntrico}
    \begin{wrapfigure}{r}{5cm}
        \vspace{-20pt}
        \includegraphics[width=\linewidth]{figuras/área-de-cuadrilátero-bicéntrico/main.pdf}
    \end{wrapfigure}
    El área de una región cuadrangular determinado por un cuadrilátero inscrito a una circunferencia y circunscrito a otra circunferencia es igual a la raíz cuadrada del producto de las longitudes de los lados.

    \vspace{3cm}

    \subsection{Área de cuadrilátero circunscrito}
    \begin{wrapfigure}{r}{5cm}
        \includegraphics[width=\linewidth]{figuras/área-de-cuadrilátero-circunscrito/main.pdf}
    \end{wrapfigure}
    El área de una región cuadrangular determinado por un cuadrilátero circunscrito a una circunferencia es igual al producto del semiperímetro del cuadrilátero y el radio de la circunferencia inscrita.

\end{multicols*}
\end{document}
