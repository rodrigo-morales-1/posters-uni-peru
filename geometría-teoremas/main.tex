\documentclass{article}

\usepackage{xcolor}
\colorlet{FilledSurface}{blue!20}
\colorlet{FilledSurfaceGroupOne}{blue!20}
\colorlet{FilledSurfaceGroupTwo}{red!20}
\colorlet{FilledSurfaceGroupThree}{green!20}
\colorlet{FilledSurfaceGroupFour}{magenta!20}
\colorlet{FormulaBackground}{green!10}
\colorlet{FormulaFrame}{green}

\usepackage{graphicx}
\usepackage{multicolrule}
\columnseprule=0.5pt
\usepackage[
    a2paper,
    top=0.5cm,
    left=0.5cm,
    right=0.5cm,
    bottom=0.5cm
]{geometry}
\usepackage{wrapfig}
\usepackage{parskip}
\usepackage{tcolorbox}
\usepackage{amsmath}

\newtcbox{\myformula}{
    colback=FormulaBackground, % Sets the background color
    colframe=FormulaFrame,     % Sets the border frame color
    boxrule=1pt,             % Defines the thickness of the outer border frame
    arc=2pt,                   % Sets the radius for the rounded corners
    boxsep=2pt,                % Sets internal padding between the frame and the "left/right/top/bottom" area
    left=1pt,                  % Extra horizontal space on the left of the content
    right=1pt,                 % Extra horizontal space on the right of the content
    top=1pt,                   % Extra vertical space above the content
    bottom=1pt                 % Extra vertical space below the content
}

\newcommand\myPolygonArea[1]{\left[#1\right]}

\begin{document}
\begin{multicols*}{3}
    \section{Área de regiones cuadrángulares}

    \begin{minipage}{0.5\linewidth}
        \begin{center}
            Área de cuadrilátero bicéntrico
        \end{center}
        \includegraphics[width=\linewidth]{figuras/área-de-cuadrilátero-bicéntrico/main.pdf}
        \begin{center}
            \myformula{$\left[ABCD\right] = \sqrt{abcd}$}
        \end{center}
    \end{minipage}
    \begin{minipage}{0.5\linewidth}
        \begin{center}
            Área de cuadrilátero circunscrito
        \end{center}
        \includegraphics[width=\linewidth]{figuras/área-de-cuadrilátero-circunscrito/main.pdf}
        \begin{center}
            \myformula{$\left[ABCD\right] = p \cdot r$}
        \end{center}
        Semiperímetro: $p = \dfrac{a+b+c+d}{2}$
    \end{minipage}

    \rule{\linewidth}{0.4pt}

    \begin{minipage}{0.5\linewidth}
        \begin{center}
            Área de cuadrilátero inscrito
        \end{center}
        \vspace{10pt}
        \includegraphics[width=\linewidth]{figuras/área-de-cuadrilátero-inscrito/main.pdf}
        \myformula{$\left[ABCD\right] = \sqrt{\left(p - a\right) \left(p - b\right) \left(p - c\right) \left(p - d\right)}$}
    \end{minipage}
    \begin{minipage}{0.5\linewidth}
        \includegraphics[width=\linewidth]{figuras/área-de-cuadrilátero-usando-diagonales/main.pdf}
        \begin{center}
            \myformula{$\left[ABCD\right] = \dfrac{AC \cdot BD}{2} \sin\left(\alpha\right)$}
        \end{center}
    \end{minipage}

    \section{Relación de áreas en regiones cuadrangulares}

    \begin{minipage}{0.5\linewidth}
        \includegraphics[width=\linewidth]{figuras/relación-de-áreas-en-regiones-cuadrangulares-producto-de-2-áreas/main.pdf}
        \begin{center}
            \myformula{$S_1 + S_3 = S_2 + S_4$}
        \end{center}
    \end{minipage}
    \begin{minipage}{0.5\linewidth}
        \includegraphics[width=\linewidth]{figuras/relación-de-áreas-en-regiones-cuadrangulares-área-determinada-por-segmentos-puntos-medios/main.pdf}
        \begin{center}
            \myformula{$\myPolygonArea{MNPQ} = \dfrac{1}{2} \myPolygonArea{ABCD}$}
        \end{center}
    \end{minipage}

    \begin{minipage}{0.5\linewidth}
        \includegraphics[width=\linewidth]{figuras/relación-de-áreas-en-regiones-cuadrangulares-producto-de-2-áreas/main.pdf}
        \begin{center}
            \myformula{$S_1 \cdot S_3 = S_2 \cdot S_4$}
        \end{center}
    \end{minipage}
    \begin{minipage}{0.5\linewidth}
        \includegraphics[width=\linewidth]{figuras/relación-de-áreas-en-regiones-cuadrangulares-puntos-medios-y-ángulo-entre-diagonales/main.pdf}
        \begin{center}
            \myformula{$\myPolygonArea{ABCD} = MP \cdot NQ \cdot \sin\left(\alpha\right)$}
        \end{center}
    \end{minipage}

    \begin{minipage}{0.5\linewidth}
        \includegraphics[width=\linewidth]{figuras/relación-de-áreas-en-regiones-cuadrangulares-vertices-a-2-puntos-medios/main.pdf}
        \begin{center}
            \myformula{$\myPolygonArea{AMCN} = \dfrac{\myPolygonArea{ABCD}}{2}$}
        \end{center}
    \end{minipage}
    \begin{minipage}{0.5\linewidth}
        \includegraphics[width=\linewidth]{figuras/relación-de-áreas-en-regiones-cuadrangulares-vertices-a-4-puntos-medios/main.pdf}
        \begin{center}
            \myformula{$S_1 + S_3 = S_2 + S_4$}
        \end{center}
    \end{minipage}

    \begin{minipage}{0.5\linewidth}
        \includegraphics[width=\linewidth]{figuras/relación-de-áreas-en-regiones-cuadrangulares-área-de-cuadrilátero-interior-determinado-por-medianas/main.pdf}
        \begin{center}
            \myformula{$S = S_1 + S_2 + S_3 + S_4$}
        \end{center}
    \end{minipage}
    \begin{minipage}{0.5\linewidth}
        \includegraphics[width=\linewidth]{figuras/relación-de-áreas-en-regiones-cuadrangulares-suma-de-áreas-determinadas-por-segmentos-puntos-medios/main.pdf}
        \begin{center}
            \myformula{$S_1 + S_3 = S_2 + S_4$}
        \end{center}
    \end{minipage}

\end{multicols*}
\end{document}


%% Template
%% \begin{minipage}{0.5\linewidth}
%%     \includegraphics[width=\linewidth]{example-image-a}
%%     \begin{center}
%%         \myformula{$1 = 1$}
%%     \end{center}
%% \end{minipage}
%% \begin{minipage}{0.5\linewidth}
%%     \includegraphics[width=\linewidth]{example-image-a}
%%     \begin{center}
%%         \myformula{$1 = 1$}
%%     \end{center}
%% \end{minipage}
